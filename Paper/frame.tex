% !TEX root = main.tex

\section*{Framing}

Our main inspiration for the activity we propose, comes from the idea that nowadays there are several jobs that are, wanted or not, surrounded by music, often by a beat only (for simplicity let's picture some sort of mechanical industry where the active machines produce the typical constant robotic noise). We refer to the employers who are enforced to do their work, almost enforced to listen to this repeating tempo. Most of the time we know that in this kind of industry the kind of job is repetitive. For this reason we thought to divide our activities in two main tasks: the first one,\textit{Balloon Tapper} where the user is asked to tap a balloon that dynamically changes position. In our oponion this could easily represent the kind of job that deals with the daily use of a computer where the interaction throughout a software is the main task. The second one,\textit{Balloon Inflater}, where the user has to inflate a balloon simply producing an action that is almost static, since the balloon doesn't move, pushing the user to statically tap the same part of the screen. We imagined this scenario could instead refer to the kind of job that employers do in a production chain (catena di montaggio?) where they are only asked to perform a static,constant and repetitive task. 
We wonder how, in this scenario, these people could possibly be affected in terms of behavior(performance?), more particularly if they would synchronize with the offered beat or otherwise, they would totally remain extraneous.
Moreover, in future, we suppose it would be really interesting to go deeper in this analysis, trying to address the goal in investigating if the beat would be so powerful that the "listeners" would speed up or slow down their activity. We believe that this would be a really big discover that could change the way to organize the usual work environment,


