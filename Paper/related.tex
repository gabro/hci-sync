% !TEX root = main.tex
\section{Related work}

The influence of music tempo can be observed and verified in many aspects. In \cite{mcelrea:drinking}, the result that the faster background music tempo can decrease the speed of drinking wine significantly is observed. Also, in \cite{patel:non-human1}, a close relation between background music tempo and driving behaviour is indicated: faster music tempo increases driving speed. Also, in the higher tempo background music, the behaviour of traffic violation such as disregarding traffic signals, lane crossing and collisions in simulated environment increased. 

As for the synchronization with music tempo, an experiment on a non-human animal, cockatoo, is conducted by \cite{patel:non-human1}, \cite{patel:non-human2}, \cite{patel:non-human3}. It shows that animals can also have the potential to synchronize with music tempo by nodding with the beat of music tempo. To verify their hypothesis, certain data were collected. They manipulated the correlation between the inter-bob-interval of cockatoo and the music tempo, made the statistics on the time cockatoo synchronizing with tempo and percentage of synchronizing time, percentage of nodding synchronizing with tempo, counts of bouts in which cockatoo synchronizing with tempo in each trial of different tempo speed. Also, they discovered that cockatoo does not always synchronize with tempo but only in discontinuous period, so they observed the data only valuable within those synchronizing periods to find the pattern of synchronization.

Also, not only background music tempo, but also music itself can have great impact on behaviors as well, which can be our guidance of project. In \cite{areni:shopping}, more appearance of classical music in wine store can encourage customers' preference of purchasing more expensive wines. In \cite{Caldwell:dining}, a experiment indicated that background music preference, rather than music tempo, can have an impact on the customers' dining time. 

All of these papers above mainly focus on the influence of background music. However, none of them have discussed about the potential influence of the music tempo on the human-computer interaction exactly. Therefore, we try to discover the possible relationship between music tempo and human input behavior on portable devices with touch screen.
