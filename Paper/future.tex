% !TEX root = main.tex

\section{Future development}\label{sec:future}
As we discussed in the \nameref{sec:conclusions} section we identified a convolution of three variables that may have affected the final results of our study.
We will now discuss which studies could be conducted in order to separate and weight the importance of each single variable.

In order to understand the effect of the tempo's BPM would interesting to investigate whether a synchronization behavior can be reproduced with different BPMs and, if so, at which extent.
For instance we noticed that users synchronize at a 130 BPM tempo while performing the \testfirst activity, but we don't know whether they would to the same with a faster or slower tempo.
It is likely that there may exist one or multiple windows of BPMs for which a human subject tend to synchronize with the background music tempo, but it clearly needs to be extensively tested with actual data.
Setting the bounds of the effect we highlighted in our study could also help in applying it to real life situation, in which, as stated in the \nameref{sec:framing} section, may be useful to enforce a slower or faster behavior, e.g. in as video-games and work environments.

In order to understand instead the effect of the cognitive workload required an interesing experiment to conduct would be to variate it, keeping the other variables constant. For instance it could possible to set up an activity in which the cognitive workload required  is initially very low (like a tapping activity with a very big balloon, difficult to miss) and gradually increase it (for instance shrinking the size of the balloon).

A similar approach could be useful for investingating the effects of the presence of a implicit or explicit goal. One could set up an activity in which there's no goal at the beginning, \testfirst is a good example, and then gradually increase the goal requirements, for instance assigning an explicit score according to the number of performed taps or their precision.
Such an experiment could help in discovering whether the user tends to ignore the background music as soon as the goal requirements become more strict and evident.