% !TEX root = main.tex
\section*{Abstract}

Music has  always represented an important factor in human lives. As the famous philosopher Friedrich Nietzsche said, ``Without music life would be a mistake''. Our body often seems to confirm this theory. In fact, during a normal daily routine we sometimes have to execute tasks while we are forced to listen to music coming from radios, computers, smartphones, etc. What we notice in these particular situations is that most of the times, in several activities, our body gestures end up trying to synchronize with the background music. With this paper we present data from an experimental study on synchronization to music. The study is based on an application for tablet devices that we have implemented. It is divided in two trivial activities: the first one, \emph{\testfirst}, where the user has to tap a balloon that changes position in the screen every time is hit; the second one, \emph{\testsecond}, where the user has to constantly tap a balloon in order to inflate it until, at some point, it explodes. Both of the tasks have to be performed  with and without music. Our main goal is to understand whether or not the user turns out to synchronize the performance of the two simple tasks with the offered music. Moreover, we want to analyze if the user is not only synchronizing with the music but even forced to slow down the performance because of the beat.