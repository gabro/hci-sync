% !TEX root = main.tex
\section{Conclusions}
\label{sec:conclusions}
We will now summarize the meaning of the results, that we already partially discussed during their presentation.

The consistent regularity, the good synchronization score and the synchronizatio ratio close to 1 indicate that the first test, \testfirst, enforces a good synchronization with the background tempo.

On the other hand an irregular tapping behavior, along with a poor synchronization score and a synchronization ratio far both of 0.68 (and therefore both far from the period, 1, and the half period, 0.5) indicate that in the second test, \testsecond, the background music is not affecting the behavior of the users in an appreciable way.

Thus, the synchronization is highly influenced by the kind of task and/or the background tempo. So far we identified three major components representing the main differences between the two experiments:

\begin{description}
	\item[Cognitive workload] While the \testfirst experiment required the user to stay focused on the activity in order to follow the balloon on the screen, the \testsecond experiment did not require any focus: one could have successfully completed the activity without even looking at the screen.

	Therefore theres is a well denoted difference between the two experiments in terms of cognitive workload required to perform the proposed activities.
straight-tunnel steering
	\item[Goal-driven] The \testfirst experiment was a potentially never-ending activity with a very regular and plain evolution (a balloon moving randomly around the screen without variating in shape or color). Therefore the user did not perceive any goal to reach neither implicit nor explicit. The \testsecond experiment, on the other hand, had an implicit goal-driven fashion. In fact, even if the users were not aware of the final explosion of the balloon, its inflation behavior suggested an evolution of the activity towards a critical point.

	This is another significant difference between the two activities

	\item[Tempo] In order to choose the BPM for each of the experiments we applied arbitrary variations to the natural BPM recorded in the bases case, as already discussed in Section~\ref{sec:basecases}. Specifically we decided to speed up the tempo for the \testfirst experiment and to slow it down for the \testsecond experiment.
	Such arbitrary decision introduced a difference between the two tests and therefore has to be accounted as a possible variable affecting the final results. 
\end{description}

Given our study it is not possible to discriminate and understand the importance of such variables over the final results that we presented above.

We will then discuss in Section~\ref{sec:future} which studies could be conducted in order to separate and weight the importance of each single variable, but all we can say so far is that the synchronization is probably affected by a convolution of those three variables.